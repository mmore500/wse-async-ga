\section{Conclusion} \label{sec:conclusion}

Proposed work lays foundations to harness wafer-scale computing for agent-based modeling.
Orders-of-magnitude scale-up will bring entirely new classes of cross-scale dynamics within reach across many ABM application areas.
To advance on this front, we explore a fundamental re-frame of simulation that shifts from a paradigm of complete, perfect data observability to dynamic, approximate observability akin to estimation approaches traditionally used to study physical systems.

Within ABM, our proposed work focuses in particular on the topic of evolutionary computation.
We will harness cutting-edge \textit{hereditary stratigraphy} (HStrat) approaches to distributed phylogenetic tracking to investigate fundamental questions about the relationship between phylogenetic structure, population scale, and time scale.
These new HStrat methods, and extensive development targeting the Cerebras CSL using the SDK hardware simulator, position our project to hit the ground running.
Our project design emphasizes explicit steps to demonstrate and evaluate novel components of underlying implementation in order to form technical foundations for ongoing work in this vein.
We are committed to organizing project implementation to produce reusable CSL and Python software tools to catalyze research projects in this area among the broader ABM community.

Potential extensions of this work include repurposing of HStrat's rolling cross-temporal sampling algorithms to facilitate efficient downsampled observability in additional application domains.
Possibilities include sampling time series activity at simulation sites or extracting coarsened agent histories (e.g., position over time).
Within the domain of evolution modeling, we anticipate building from work proposed here to conduct cross-scale evo-epidemiological research into how pathogen trait evolution confronts fitness trade-offs related to within-host infection dynamics and between-host transmissibility.
