\section{Objectives} \label{sec:objectives}

With access to CS-2 hardware, we plan to evaluate the viability of reconstruction-based methodology for \textit{in silico} phylogenetic tracking at scale in production and then incorporate it within large-scale epidemiological modeling experiments performed via a custom CSL kernel.
Technical demonstration and evaluation of methodology stitched together into answering technical questions about huge-scale phylogenetic structure that existing approaches have yet been able to address.

Application areas are general evolutionary theory/understanding, which has applications in many sectors including ecology, natural history, application-oriented evolutionary computation.
Prelude to more targeted work in evolutionary epidemiology.

\subsection{Technology Demonstration and Scaling Evaluation}
Objective one: technology demonstration of reconstruction at wafer scale.
Validate by running under two simple evolutionary conditions that expect to produce different phylogenetic structure.
See if we can recover that phylogenetic structure.

Also, observe scalability of asynchronous evolutionary approach.

\subsection{Effects of Scale in time and space on Phylogenetic Structure}
Objective two: are phylogenetic structure metrics scale invariant?
Run several evolutionary conditions with small colletions of PEs and large collections of PEs
Test for differences in phylogenetic structure metrics.
(how to control for reconstruction quality?)

Relatedly: are phylometrics time invariant to duration of simulation?
Another objective: does streaming samples off the chip asynchronously bias phylometrics (as opposed to collecting everything all at the end.


% Objective three: what is the relationship between individual phylogenies and species-level phylogenies?
