\section{Preliminary Work} \label{sec:preliminary-work}

In preparation for proposed work, we joined the Cerebras SDK program \citep{selig2022cerebras} and assembled Cerebras Software Language (CSL) software implementations that will be necessary to conduct experiments with WSE hardware.
For the time being, we have used Cerebras' hardware emulator to test our software implementations.
This section reports a small-scale experiment performed to validate the core software functionality.
In addition to this integration test, software quality has also been verified through an associated unit test suite.

\subsection{Validation Experiment}

% fab w,h = 10,5
% Kernel x,y w,h = 4,1 3,3
% memcpy x,y w,h = 1,1 8,3
% whoami ===============================================================
% [[0 3 6]
%  [1 4 7]
%  [2 5 8]]
% whereami x ===========================================================
% [[0 1 2]
%  [0 1 2]
%  [0 1 2]]
% whereami y ===========================================================
% [[0 0 0]
%  [1 1 1]
%  [2 2 2]]
% cycle counter =======================================================
% [[120 120 120]
%  [120 120 120]
%  [120 120 120]]
% recv counter N ========================================================
% [[ 1  1  1]
%  [33 31 29]
%  [29 28 29]]
% recv counter S ========================================================
% [[31 33 31]
%  [33 31 29]
%  [ 1  1  1]]
% recv counter E ========================================================
% [[31 29  1]
%  [28 29  1]
%  [30 33  1]]
% recv counter W ========================================================
% [[ 1 28 29]
%  [ 1 31 32]
%  [ 1 33 30]]
% send counter N ========================================================
% [[  0   0   0]
%  [124 130 124]
%  [132 122 114]]
% send counter S ========================================================
% [[130 126 112]
%  [118 112 118]
%  [  0   0   0]]
% send counter E ========================================================
% [[112 116   0]
%  [124 130   0]
%  [132 122   0]]
% send counter W ========================================================
% [[  0 126 112]
%  [  0 112 118]
%  [  0 120 128]]

In addition to unit and integration tests, we performed a simple experiment to validate generated phylogenetic reconstructions.
In this experiment, we evolved a population of simple genomes with HStrat annotations under drift conditions on the island model GA kernel, exported end-state genomes for postprocessing, and used the HStrat annotations to reconstruct their phylogenetic history.

Simulation ran for 120 PE generation cycles with a tournament size of 5 and a population size of 32 per PE.
For simplicity, drift conditions were imposed --- all genomes were evaluated to a fitness value of 0.0.
Due to slow execution of the hardware simulator, we performed this experiment using a small 3x3 grid of processor elements.

Genomes were fixed-length 3-word arrays represented using data type \texttt{u32}.
Figure \ref{fig:validation-example:genomes} details the content and layout of genomes, and provides example genome values yielded from simulation.
Notably, the first sixteen bits were used to tag clade geneses.
At the outset of simulation, founding population members were each assigned a randomized tag value.
This value was inherited without mutation throughout simulation.
Thus, it can be used to identify end-state genomes that evolved from the same original ancestor.

The send buffers were sized to hold one genome and the receive buffers to hold 4 genomes.
Within the simulation window, between 28 and 33 receive cycles were completed for each PE and between 112 and 132 send cycles were completed for each PE.

\begin{figure}[htbp]
    \centering

    % Top panel for the image
    \begin{subfigure}[b]{\textwidth}
        asdf
        % \includegraphics[width=\textwidth]{path/to/your/image.jpg}
        \caption{Caption for the image}
    \end{subfigure}

    % Space between image and table
    \vspace{10pt} 

    % Bottom panel for the table
    \begin{subfigure}[b]{\textwidth}
        \centering
    \begin{tabular}{
    c
    >{\columncolor{LighterBlue}}c
    >{\columncolor{LighterBlue}}c
    >{\columncolor{LighterSalmon}}c
    >{\columncolor{LighterSalmon}}c
    q % Thick divider
    >{\columncolor{LighterPastelGreenYellow}}c
    >{\columncolor{LighterPastelGreenYellow}}c
    >{\columncolor{LighterPastelGreenYellow}}c
    >{\columncolor{LighterPastelGreenYellow}}c
    q % Thick divider
    >{\columncolor{LighterPastelGreenYellow}}c
    >{\columncolor{LighterPastelGreenYellow}}c
    >{\columncolor{LighterPastelGreenYellow}}c
    >{\columncolor{LighterPastelGreenYellow}}c
    }
    & \multicolumn{4}{cq}{\cellcolor{white}Word 0} & \multicolumn{4}{cq}{\cellcolor{white}Word 1} & \multicolumn{4}{c}{\cellcolor{white}Word 2} \\
    \cmidrule(l{1.5pt}r{1.5pt}){2-5}
    \cmidrule(l{1.5pt}r{1.5pt}){6-9}
    \cmidrule(l{1.5pt}r{1.5pt}){10-13}
    & {\cellcolor{white}B 0} & {\cellcolor{white}B 1} & {\cellcolor{white}B 2} & {\cellcolor{white}B 3} & {\cellcolor{white}B 4} & {\cellcolor{white}B 5} & {\cellcolor{white}B 6} & {\cellcolor{white}B 7} & {\cellcolor{white}B 8} & {\cellcolor{white}B 9} & {\cellcolor{white}B 1}0 & {\cellcolor{white}B 1}1 \\
    \cmidrule(l{1.5pt}r{1.5pt}){2-2}
    \cmidrule(l{1.5pt}r{1.5pt}){3-3}
    \cmidrule(l{1.5pt}r{1.5pt}){4-4}
    \cmidrule(l{1.5pt}r{1.5pt}){5-5}
    \cmidrule(l{1.5pt}r{1.5pt}){6-6}
    \cmidrule(l{1.5pt}r{1.5pt}){7-7}
    \cmidrule(l{1.5pt}r{1.5pt}){8-8}
    \cmidrule(l{1.5pt}r{1.5pt}){9-9}
    \cmidrule(l{1.5pt}r{1.5pt}){10-10}
    \cmidrule(l{1.5pt}r{1.5pt}){11-11}
    \cmidrule(l{1.5pt}r{1.5pt}){12-12}
    \cmidrule(l{1.5pt}r{1.5pt}){13-13}
    & \multicolumn{4}{cq}{\cellcolor{white}} & \multicolumn{4}{cq}{\cellcolor{white}} & \multicolumn{4}{c}{\cellcolor{white}} \\[-2ex]
    Genome 1 & \texttt{F9} & \texttt{02} & \texttt{79} & \texttt{00} & \texttt{8D} & \texttt{22} & \texttt{4F} & \texttt{F3} & \texttt{D2} & \texttt{78} & \texttt{AD} & \texttt{C7} \\
    & \multicolumn{4}{cq}{\cellcolor{white}} & \multicolumn{4}{cq}{\cellcolor{white}} & \multicolumn{4}{c}{\cellcolor{white}} \\[-2ex]
    Genome 2 & \texttt{F9} & \texttt{02} & \texttt{75} & \texttt{00} & \texttt{8D} & \texttt{A1} & \texttt{CB} & \texttt{F2} & \texttt{D1} & \texttt{5B} & \texttt{CC} & \texttt{D4} \\
    & \multicolumn{4}{cq}{\cellcolor{white}} & \multicolumn{4}{cq}{\cellcolor{white}} & \multicolumn{4}{c}{\cellcolor{white}} \\[-2ex]
    Genome 3 & \texttt{61} & \texttt{B6} & \texttt{65} & \texttt{00} & \texttt{66} & \texttt{29} & \texttt{B4} & \texttt{F0} & \texttt{62} & \texttt{99} & \texttt{5A} & \texttt{61} \\
    {\cellcolor{white}\ldots} & {\cellcolor{white}\ldots} & {\cellcolor{white}\ldots} & {\cellcolor{white}\ldots} & {\cellcolor{white}\ldots} & {\cellcolor{white}\ldots} & {\cellcolor{white}\ldots} & {\cellcolor{white}\ldots} & {\cellcolor{white}\ldots} & {\cellcolor{white}\ldots} & {\cellcolor{white}\ldots} & {\cellcolor{white}\ldots} \\
    \end{tabular}

        \caption{Caption for the table}
    \end{subfigure}

    \caption{Caption for the entire figure}
    \label{fig:my_figure}
\end{figure}



    % \begin{tabular}{
    % c
    % >{\columncolor{LighterBlue}}c
    % >{\columncolor{LighterBlue}}c
    % >{\columncolor{LighterSalmon}}c
    % >{\columncolor{LighterSalmon}}c
    % q % Thick divider
    % >{\columncolor{LighterPastelGreenYellow}}c
    % >{\columncolor{LighterPastelGreenYellow}}c
    % >{\columncolor{LighterPastelGreenYellow}}c
    % >{\columncolor{LighterPastelGreenYellow}}c
    % q % Thick divider
    % >{\columncolor{LighterPastelGreenYellow}}c
    % >{\columncolor{LighterPastelGreenYellow}}c
    % >{\columncolor{LighterPastelGreenYellow}}c
    % >{\columncolor{LighterPastelGreenYellow}}c
    % }
    % & \multicolumn{4}{cq}{\cellcolor{white}Word 0} & \multicolumn{4}{cq}{\cellcolor{white}Word 1} & \multicolumn{4}{c}{\cellcolor{white}Word 2} \\
    % \cmidrule(l{1.5pt}r{1.5pt}){2-5}
    % \cmidrule(l{1.5pt}r{1.5pt}){6-9}
    % \cmidrule(l{1.5pt}r{1.5pt}){10-13}
    % & {\cellcolor{white}B 0} & {\cellcolor{white}B 1} & {\cellcolor{white}B 2} & {\cellcolor{white}B 3} & {\cellcolor{white}B 4} & {\cellcolor{white}B 5} & {\cellcolor{white}B 6} & {\cellcolor{white}B 7} & {\cellcolor{white}B 8} & {\cellcolor{white}B 9} & {\cellcolor{white}B 1}0 & {\cellcolor{white}B 1}1 \\
    % \cmidrule(l{1.5pt}r{1.5pt}){2-2}
    % \cmidrule(l{1.5pt}r{1.5pt}){3-3}
    % \cmidrule(l{1.5pt}r{1.5pt}){4-4}
    % \cmidrule(l{1.5pt}r{1.5pt}){5-5}
    % \cmidrule(l{1.5pt}r{1.5pt}){6-6}
    % \cmidrule(l{1.5pt}r{1.5pt}){7-7}
    % \cmidrule(l{1.5pt}r{1.5pt}){8-8}
    % \cmidrule(l{1.5pt}r{1.5pt}){9-9}
    % \cmidrule(l{1.5pt}r{1.5pt}){10-10}
    % \cmidrule(l{1.5pt}r{1.5pt}){11-11}
    % \cmidrule(l{1.5pt}r{1.5pt}){12-12}
    % \cmidrule(l{1.5pt}r{1.5pt}){13-13}
    % & \multicolumn{4}{cq}{\cellcolor{white}} & \multicolumn{4}{cq}{\cellcolor{white}} & \multicolumn{4}{c}{\cellcolor{white}} \\[-2ex]
    % Genome 1 & \texttt{F9} & \texttt{02} & \texttt{79} & \texttt{00} & \texttt{8D} & \texttt{22} & \texttt{4F} & \texttt{F3} & \texttt{D2} & \texttt{78} & \texttt{AD} & \texttt{C7} \\
    % & \multicolumn{4}{cq}{\cellcolor{white}} & \multicolumn{4}{cq}{\cellcolor{white}} & \multicolumn{4}{c}{\cellcolor{white}} \\[-2ex]
    % Genome 2 & \texttt{F9} & \texttt{02} & \texttt{75} & \texttt{00} & \texttt{8D} & \texttt{A1} & \texttt{CB} & \texttt{F2} & \texttt{D1} & \texttt{5B} & \texttt{CC} & \texttt{D4} \\
    % & \multicolumn{4}{cq}{\cellcolor{white}} & \multicolumn{4}{cq}{\cellcolor{white}} & \multicolumn{4}{c}{\cellcolor{white}} \\[-2ex]
    % Genome 3 & \texttt{61} & \texttt{B6} & \texttt{65} & \texttt{00} & \texttt{66} & \texttt{29} & \texttt{B4} & \texttt{F0} & \texttt{62} & \texttt{99} & \texttt{5A} & \texttt{61} \\
    % {\cellcolor{white}\ldots} & {\cellcolor{white}\ldots} & {\cellcolor{white}\ldots} & {\cellcolor{white}\ldots} & {\cellcolor{white}\ldots} & {\cellcolor{white}\ldots} & {\cellcolor{white}\ldots} & {\cellcolor{white}\ldots} & {\cellcolor{white}\ldots} & {\cellcolor{white}\ldots} & {\cellcolor{white}\ldots} & {\cellcolor{white}\ldots} \\
    % \end{tabular}



At the end of simulation, one genome was sampled per PE.
Figure \ref{fig:validation-example:phylogeny} shows the result of phylogenetic reconstruction on these sampled genomes.
To simplify visualization of the phylogeny with existing phyloinformatic tools, all clades were stitched together into a single tree by attaching them to a common ancestor.
Taxa are colored according to their clade genesis tag.
Taxon arrangement in the reconstructed tree agrees well with founding clade identities.
All different-colored taxa are predicted to have had ancient common ancestry.
Note that relatedness between \texttt{0x5fb} and \texttt{0xde40} is slightly overestimated, due to a configured lack of precise resolution necessary to ensure compact genetic annotations.
Applications requiring high reconstruction precision can opt to use larger HStrat annotations.

\subsection{Software and Data Availability}

Reported work used CSL code hosted on GitHub at \url{https://github.com/mmore500/wse-sketches/tree/v0.1.0} \citep{moreno2024wse} with the Cerebras SDK v1.0.0 compiler and hardware emulator \citep{selig2022cerebras}.
Reference Python implementations for HStrat surface algorithms are v0.3.0 at \citep{moreno2024hsurf}.
Reconstruction used the Python \texttt{hstrat} package, v1.11.1 \citep{moreno2022hstrat}.
Data from this experiment is available via the Open Science Framework at \url{https://osf.io/bfm2z/} \citep{moreno2024toward} and associated notebook code can be found at \url{https://hopth.ru/cm}.
